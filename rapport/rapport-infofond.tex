\documentclass[a4paper,11pt]{article}

\usepackage[french]{babel}
\usepackage[utf8]{inputenc}
\usepackage[left=2.5cm,top=2cm,right=2.5cm,nohead,nofoot]{geometry}
\usepackage{url}
\usepackage{graphicx}
\usepackage{hyperref}
\usepackage{listings}
\usepackage{amsmath}
\usepackage{amssymb}
\usepackage{color}



\linespread{1.1}



\begin{document}

\begin{titlepage}
\begin{center}
\textbf{\textsc{UNIVERSIT\'E LIBRE DE BRUXELLES}}\\
%\textbf{\textsc{Faculté des Sciences}}\\
%\textbf{\textsc{Département d'Informatique}}
\vfill{}\vfill{}
\begin{center}{\Huge Projet : Logique Propositionnelle et Utilisation de l’Outil MiniSat}\end{center}{\Huge \par}
\begin{center}{\large Pierre Gérard, Antoine Carpentier}\end{center}{\Huge \par}
\vfill{}\vfill{} \vfill{}
\begin{flushleft}{\large \textbf{INFO-F-302 Informatique Fondamentale}}\hfill{Emmanuel FILIOT, Guillermo Pérez}\end{flushleft}{\large\par}
\vfill{}\vfill{}\enlargethispage{3cm}
\textbf{Année académique 2014~-~2015}
\end{center}
\end{titlepage}

%\begin{abstract}
%Ce rapport présente ...
%\end{abstract}


\tableofcontents

\pagebreak


\section{Q1}
Les contraintes sont les suivantes :
\begin{itemize}
  \item Le nombre d'étudiant dans une salle ne peut pas dépasser sa capacité,
  \item Un étudiant ne peut pas avoir deux examens au même moment,
  \item Un professeur ne peut pas avoir deux examens au même moment,
  \item Un examen doit avoir au moins un professeur,
  \item Un examen doit avoir au moins un étudiant
  \item Un examen a au plus un professeur,
  \item Chaque examen doit se dérouler une seule fois,
  \item Chaque examen doit se dérouler dans une seule salle,
  \item Dans une salle, il ne peut se déroule qu'un seul examen a la fois.
\end{itemize}


\section{Q2}
\subsection {Le nombre d'étudiant dans une salle ne peut pas dépasser sa capacité}
\begin{displaymath}
\forall x \in X , \forall s \in S ,\forall e \in E, \forall t \in \{1,...,T\} : a(e) \mapsto \{x\} \wedge \mu(x) = (s,t) \wedge \sum a(e) \mapsto \{x\} \leq c(s)
\end{displaymath}	

\subsection {Un étudiant ne peut pas avoir deux examens au même moment}
\begin{displaymath}
\forall x_{1},x_{2} \in X, \forall s_{1},s_{2} \in S , \forall e \in E ,\forall t_{1}, t_{2} \in \{1,...,T\} :  a(e) \mapsto \{x_{1},x_{2}\}  \wedge \mu(x_{1}) = (s_{1},t_{1}) \wedge \mu(x_{2}) = (s_{2},t_{2}) \wedge t_{1} != t_{2} \wedge x_{1} != x_{2}
\end{displaymath}
\subsection {Un professeur ne peut pas avoir deux examens au même moment}
\begin{displaymath}
\forall x_{1},x_{2} \in X, \forall s_{1},s_{2} \in S , \forall p \in P ,\forall t_{1}, t_{2} \in \{1,...,T\} :  b(p) \mapsto \{x_{1},x_{2}\}  \wedge \mu(x_{1}) = (s_{1},t_{1}) \wedge \mu(x_{2}) = (s_{2},t_{2}) \wedge t_{1} != t_{2} \wedge x_{1} != x_{2}
\end{displaymath}
\subsection {Un examen doit avoir au moins un professeur}
\begin{displaymath}
\forall x \in X, \exists p \in P : b(p) \mapsto \{x\} 
\end{displaymath}
\subsection {Un examen doit avoir au moins un étudiant}
\begin{displaymath}
\forall x \in X, \exists e \in E : a(e) \mapsto \{x\}
\end{displaymath}
\subsection {Un examen a au plus un professeur}
\begin{displaymath}
\forall x \in X, \nexists p_{1},p_{2} \in P: p_{1} != p_{2} \wedge b(p_{1}) \mapsto \{x\} \wedge b(p_{2}) \mapsto \{x\} 
\end{displaymath}
\subsection {Chaque examen doit se dérouler une seule fois}
\begin{displaymath}
\forall t_{1}, t_{2} \in \{1,...,T\},\forall s_{1},s_{2} \in S, \nexists x \in X : t_{1} != t_{2} \wedge \mu(x) = (s_{1},t_{1}) \wedge \mu(x) = (s_{2},t_{2})
\end{displaymath}
\subsection {Chaque examen doit se dérouler dans une seule salle}
\begin{displaymath}
\forall t_{1}, t_{2} \in \{1,...,T\},\forall s_{1},s_{2} \in S, \nexists x \in X : s_{1} != s_{2} \wedge \mu(x) = (s_{1},t_{1}) \wedge \mu(x) = (s_{2},t_{2})
\end{displaymath}	
\subsection {Dans une salle, il ne peut se dérouler qu'un seul examen a la fois}
\begin{displaymath}
\forall t_{1}, t_{2} \in \{1,...,T\},\forall x_{1},x_{2} \in X, \nexists s \in S : t_{1} = t_{2} \wedge x_{1} != x_{2} \wedge \mu(x_{1}) = (s,t_{1}) \wedge \mu(x_{2}) = (s,t_{2})
\end{displaymath}	

\section{Q3}
Pour construire la formule en FNC, nous avons besoins de définir de nouvelle variable.
\begin{itemize}
    \item \(E\) est le nombre d'étudiants
    \item \(P\) est le nombre de professeurs
    \item \(X\) est le nombre d'examens
    \item \(S\) est le nombre de salles
    \item \(T\) est le nombre de périodes de temps
	\item \( A_{e,x}\) telle que l'étudiant e passe l'examen x,
	\item \(B_{p,x}\) telle que le professeur p donne l'examen x ,
	\item \(C_{s,i}\) telle que la salle s a la capacité d'accueillir i étudiants,
	\item \(M_{x,s,t}\) telle que l'examen x se déroule dans la salle s au temps t,
	\item \(N_{x}\) telle que la fonction N donne le nombre d'étudiant qui passe l'examen x.
\end{itemize}


\subsection{Ensemble de FNC}
(p1)
\begin{displaymath}
	\bigwedge\limits_{t=1}^{T}\bigwedge\limits_{x=1}^{X}\bigwedge\limits_{\substack{s=1 \\ M_{x,s,t}}}^{S} C_{s,N_{x}}
\end{displaymath}

(p2)
\begin{displaymath}
\bigwedge\limits_{t=1}^{T}\bigwedge\limits_{s=1}^{S}\bigwedge\limits_{e=1}^{E}\bigwedge\limits_{\substack{i=1 \\ A_{e,i}}}^{X}\bigwedge\limits_{j=1}^{i} \neg M_{i, s, t} \vee \neg M_{j, s, t}
\end{displaymath}

(p3)
\begin{displaymath}
\bigwedge\limits_{t=1}^{T}\bigwedge\limits_{s=1}^{S}\bigwedge\limits_{p=1}^{P}\bigwedge\limits_{\substack{i=1 \\ B_{p,x}}}^{X}\bigwedge\limits_{j=1}^{i} \neg M_{i, s, t} \vee \neg M_{j, s, t}
\end{displaymath}

(p4)
\begin{displaymath}
\bigwedge\limits_{x=1}^{X}\bigvee\limits_{p=1}^{P} B_{p, x}
\end{displaymath}

(p5)
\begin{displaymath}
\bigwedge\limits_{x=1}^{X}\bigvee\limits_{e=1}^{E} A_{e, x}
\end{displaymath}

(p6)
\begin{displaymath}
\bigwedge\limits_{x=1}^{X}\bigwedge\limits_{i=1}^{P}\bigwedge\limits_{j=1}^{i} \neg B_{j,x} \vee \neg B_{i, x}
\end{displaymath}

(p7)
\begin{displaymath}
\bigwedge\limits_{x=1}^{X}\bigvee\limits_{t=1}^{T}\bigvee\limits_{s=1}^{S} M_{x, s, t}
\end{displaymath}
\begin{displaymath}
\bigwedge\limits_{x=1}^{X}\bigwedge\limits_{s=1}^{S}\bigwedge\limits_{i=1}^{T}\bigwedge\limits_{j=1}^{i} \neg M_{x, s, i} \vee \neg M_{x, s, j}
\end{displaymath}

(p8)
\begin{displaymath}
\bigwedge\limits_{x=1}^{X}\bigwedge\limits_{t=1}^{T}\bigwedge\limits_{i=1}^{S}\bigwedge\limits_{j=1}^{i} \neg M_{x, i, t} \vee \neg M_{x, j, t}
\end{displaymath}

(p9)
\begin{displaymath}
\bigwedge\limits_{s=1}^{S}\bigwedge\limits_{t=1}^{T}\bigwedge\limits_{i=1}^{X}\bigwedge\limits_{j=1}^{i} \neg M_{i, s, t} \vee \neg M_{j, s, t}
\end{displaymath}


\subsection{FNC final}
La forme normal conjonctive de la formule est:
\begin{displaymath}
\Phi_{I} = p1 \wedge p2 \wedge p3 \wedge p4 \wedge p5 \wedge p6 \wedge p7 \wedge p8 \wedge p9
\end{displaymath}
\section{Q10}

LES ETUDIANTS VEULENT BOUFFER LE TEMPS DE MIDI !!!!!!!!!
durée max d'un examen
le week end
overbooking sur les salles
certain examen dans certain salle (salle info)


\section{Q11}
Un problème est NP-complet ou NP-dur si il peut se ramener sous la forme d'un autre problème NP-complet. Si un problème peut se mettre en forme normal conjonctive et satisfaisable, il est SAT. Notre problème possède au moins 3 clauses, en effet TODO . Il est donc 3-SAT et donc il est aussi NP-complet donc on peut déduire que notre problème est NP-complet ou NP-dur.
\end{document}

