\documentclass[a4paper,10pt]{article}

\usepackage[french]{babel}
\usepackage[utf8]{inputenc}
\usepackage[left=2.5cm,top=2cm,right=2.5cm,nohead,nofoot]{geometry}
\usepackage{url}
\usepackage{graphicx}
\usepackage{hyperref}
\usepackage{listings}
\usepackage{amsmath}
\usepackage{amssymb}
\usepackage{color}



\linespread{1.1}



\begin{document}

\begin{titlepage}
\begin{center}
\textbf{\textsc{UNIVERSIT\'E LIBRE DE BRUXELLES}}\\
%\textbf{\textsc{Faculté des Sciences}}\\
%\textbf{\textsc{Département d'Informatique}}
\vfill{}\vfill{}
\begin{center}{\Huge Projet : Logique Propositionnelle et Utilisation de l’Outil MiniSat}\end{center}{\Huge \par}
\begin{center}{\large Pierre Gérard, Antoine Carpentier}\end{center}{\Huge \par}
\vfill{}\vfill{} \vfill{}
\begin{flushleft}{\large \textbf{INFO-F-302 Informatique Fondamentale}}\hfill{Emmanuel FILIOT, Guillermo Pérez}\end{flushleft}{\large\par}
\vfill{}\vfill{}\enlargethispage{3cm}
\textbf{Année académique 2014~-~2015}
\end{center}
\end{titlepage}

%\begin{abstract}
%Ce rapport présente ...
%\end{abstract}


\tableofcontents

\pagebreak


\section{Q1}
Les contraintes sont les suivantes :
\begin{itemize}
  \item Le nombre d'étudiant dans une salle ne peut pas dépasser sa capacité,
  \item Un étudiant ne peut pas se trouver dans deux salles au même moment,
  \item Un professeur ne peut pas se trouver dans deux salles au même moment,
  \item Un examen doit avoir au moins un professeur et au un moins un étudiant,
  \item Un examen a au plus un professeur,
  \item Chaque examen doit se dérouler une seule fois,
  \item Dans une salle, il ne peut se déroule qu'un seul examen a la fois.
\end{itemize}


\section{Q2}
\subsection {Le nombre d'étudiant dans une salle ne peut pas dépasser sa capacité}
\begin{displaymath}
\forall x \in X , \forall s \in S ,\forall e \in E, \forall t \in \{1,...,T\} : a(e) \mapsto \{x\} \wedge \mu(x) = (s,t) \wedge \sum a(e) \mapsto \{x\} \leq c(s)
\end{displaymath}	

\subsection {Un étudiant ne peut pas avoir deux examens au même moment}
\begin{displaymath}
\forall x_{1},x_{2} \in X, \forall s_{1},s_{2} \in S , \forall e \in E ,\forall t_{1}, t_{2} \in \{1,...,T\} :  a(e) \mapsto \{x_{1},x_{2}\}  \wedge \mu(x_{1}) = (s_{1},t_{1}) \wedge \mu(x_{2}) = (s_{2},t_{2}) \wedge t_{1} != t_{2} \wedge x_{1} != x_{2}
\end{displaymath}
\subsection {Un professeur ne peut pas avoir deux examens au même moment}
\begin{displaymath}
\forall x_{1},x_{2} \in X, \forall s_{1},s_{2} \in S , \forall p \in P ,\forall t_{1}, t_{2} \in \{1,...,T\} :  b(p) \mapsto \{x_{1},x_{2}\}  \wedge \mu(x_{1}) = (s_{1},t_{1}) \wedge \mu(x_{2}) = (s_{2},t_{2}) \wedge t_{1} != t_{2} \wedge x_{1} != x_{2}
\end{displaymath}
\subsection {Un examen doit avoir au moins un professeur et au un moins un étudiant}
\begin{displaymath}
\forall x \in X, \exists p \in P, \exists e \in E : a(e) \mapsto \{x\} \wedge b(p) \mapsto \{x\} 
\end{displaymath}
\subsection {Un examen a au plus un professeur}
\begin{displaymath}
\forall x \in X, \nexists p_{1},p_{2} \in P: p_{1} != p_{2} \wedge b(p_{1}) \mapsto \{x\} \wedge b(p_{2}) \mapsto \{x\} 
\end{displaymath}
\subsection {Chaque examen doit se dérouler une seule fois}
\begin{displaymath}
\forall t_{1}, t_{2} \in \{1,...,T\},\forall s_{1},s_{2} \in S, \nexists x \in X : t_{1} != t_{2} \wedge \mu(x) = (s_{1},t_{1}) \wedge \mu(x) = (s_{2},t_{2})
\end{displaymath}
\subsection {Chaque examen doit se dérouler dans une seule salle}
\begin{displaymath}
\forall t_{1}, t_{2} \in \{1,...,T\},\forall s_{1},s_{2} \in S, \nexists x \in X : s_{1} != s_{2} \wedge \mu(x) = (s_{1},t_{1}) \wedge \mu(x) = (s_{2},t_{2})
\end{displaymath}	
\subsection {Dans une salle, il ne peut se dérouler qu'un seul examen a la fois}
\begin{displaymath}
\forall t_{1}, t_{2} \in \{1,...,T\},\forall x_{1},x_{2} \in X, \nexists s \in S : t_{1} = t_{2} \wedge x_{1} != x_{2} \wedge \mu(x_{1}) = (s,t_{1}) \wedge \mu(x_{2}) = (s,t_{2})
\end{displaymath}	

\section{Q3}
Pour construire la formule en FNC, nous avons besoins de définir de nouvelle variable.
\begin{itemize}
    \item \(E\) est le nombre d'étudiants
    \item \(P\) est le nombre de professeurs
    \item \(X\) est le nombre d'examens
    \item \(S\) est le nombre de salles
    \item \(T\) est le nombre de périodes de temps
	\item \( A_{e,x}\) telle que l'étudiant e passe l'examen x,
	\item \(B_{p,x}\) telle que le professeur p donne l'examen x ,
	\item \(C_{s,i}\) telle que la salle s a la capacité d'accueillir i étudiants,
	\item \(M_{x,s,t}\) telle que l'examen x se déroule dans la salle s au temps t.
\end{itemize}

Nous avons aussi besoin
\begin{itemize}
	\item \( ALO(\varepsilon) \), au moins un élément de l'ensemble \(\varepsilon)\) est vrai,
	\item \( AMO(\varepsilon) \), au plus un élément de l'ensemble \(\varepsilon)\) est vrai.
\end{itemize}

\subsection{Formes normales conjonctives}

4
\begin{displaymath}
\bigwedge\limits_{x=1}^{X}\bigvee\limits_{p=1}^{P}\bigvee\limits_{e=1}^{E} \neg(\neg(A_{e, x}) \vee \neg(B_{p,x}))
\end{displaymath}


\end{document}