\documentclass[a4paper,10pt]{article}

\usepackage[french]{babel}
\usepackage[utf8]{inputenc}
\usepackage[left=2.5cm,top=2cm,right=2.5cm,nohead,nofoot]{geometry}
\usepackage{url}
\usepackage{graphicx}
\usepackage{hyperref}
\usepackage{listings}
\usepackage{amsmath}
\usepackage{color}



\linespread{1.1}



\begin{document}

\begin{titlepage}
\begin{center}
\textbf{\textsc{UNIVERSIT\'E LIBRE DE BRUXELLES}}\\
%\textbf{\textsc{Faculté des Sciences}}\\
%\textbf{\textsc{Département d'Informatique}}
\vfill{}\vfill{}
\begin{center}{\Huge Projet : Logique Propositionnelle et Utilisation de l’Outil MiniSat}\end{center}{\Huge \par}
\begin{center}{\large Pierre Gérard, Antoine Carpentier}\end{center}{\Huge \par}
\vfill{}\vfill{} \vfill{}
\begin{flushleft}{\large \textbf{INFO-F-302 Informatique Fondamentale}}\hfill{Emmanuel FILIOT, Guillermo Pérez}\end{flushleft}{\large\par}
\vfill{}\vfill{}\enlargethispage{3cm}
\textbf{Année académique 2014~-~2015}
\end{center}
\end{titlepage}

%\begin{abstract}
%Ce rapport présente ...
%\end{abstract}


\tableofcontents

\pagebreak


\section{Q1}
Les contraintes sont les suivantes :
\begin{itemize}
  \item Le nombre d'étudiant dans une salle ne peut pas dépasser sa capacité,
  \item Un étudiant ne peut pas se trouver dans deux salles au même moment,
  \item Un professeur ne peut pas se trouver dans deux salles au même moment,
  \item Un examen doit avoir au moins un professeur et au un moins un étudiant,
  \item Un examen a au plus un professeur,
  \item Chaque examen doit se dérouler une seule fois,
  \item Dans une salle, il ne peut se déroule qu'un seul examen a la fois.
\end{itemize}


\section{Q2}
\subsection {Le nombre d'étudiant dans une salle ne peut pas dépasser sa capacité}
\begin{displaymath}
\forall x \in X , \forall s \in S ,\forall e \in E, \forall t \in \{1,...,T\} : a(e) \mapsto {x} \wedge \mu(x) = (s,t) \wedge \sum a(e) \mapsto {x} \leq c(s)
\end{displaymath}	

\subsection {Un étudiant ne peut pas se trouver dans deux salles au même moment}

\subsection {Un professeur ne peut pas se trouver dans deux salles au même moment}

\subsection {Un examen doit avoir au moins un professeur et au un moins un étudiant}

\subsection {Un examen a au plus un professeur}

\subsection {Chaque examen doit se dérouler une seule fois}

\subsection {Dans une salle, il ne peut se déroule qu'un seul examen a la fois}




\end{document}